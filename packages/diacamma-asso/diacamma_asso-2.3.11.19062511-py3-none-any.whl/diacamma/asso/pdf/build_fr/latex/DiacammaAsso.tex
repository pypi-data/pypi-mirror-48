%% Generated by Sphinx.
\def\sphinxdocclass{report}
\documentclass[a4paper,10pt,oneside,french]{sphinxmanual}
\ifdefined\pdfpxdimen
   \let\sphinxpxdimen\pdfpxdimen\else\newdimen\sphinxpxdimen
\fi \sphinxpxdimen=.75bp\relax

\PassOptionsToPackage{warn}{textcomp}
\usepackage[utf8]{inputenc}
\ifdefined\DeclareUnicodeCharacter
% support both utf8 and utf8x syntaxes
  \ifdefined\DeclareUnicodeCharacterAsOptional
    \def\sphinxDUC#1{\DeclareUnicodeCharacter{"#1}}
  \else
    \let\sphinxDUC\DeclareUnicodeCharacter
  \fi
  \sphinxDUC{00A0}{\nobreakspace}
  \sphinxDUC{2500}{\sphinxunichar{2500}}
  \sphinxDUC{2502}{\sphinxunichar{2502}}
  \sphinxDUC{2514}{\sphinxunichar{2514}}
  \sphinxDUC{251C}{\sphinxunichar{251C}}
  \sphinxDUC{2572}{\textbackslash}
\fi
\usepackage{cmap}
\usepackage[T1]{fontenc}
\usepackage{amsmath,amssymb,amstext}
\usepackage{babel}



\usepackage{times}
\expandafter\ifx\csname T@LGR\endcsname\relax
\else
% LGR was declared as font encoding
  \substitutefont{LGR}{\rmdefault}{cmr}
  \substitutefont{LGR}{\sfdefault}{cmss}
  \substitutefont{LGR}{\ttdefault}{cmtt}
\fi
\expandafter\ifx\csname T@X2\endcsname\relax
  \expandafter\ifx\csname T@T2A\endcsname\relax
  \else
  % T2A was declared as font encoding
    \substitutefont{T2A}{\rmdefault}{cmr}
    \substitutefont{T2A}{\sfdefault}{cmss}
    \substitutefont{T2A}{\ttdefault}{cmtt}
  \fi
\else
% X2 was declared as font encoding
  \substitutefont{X2}{\rmdefault}{cmr}
  \substitutefont{X2}{\sfdefault}{cmss}
  \substitutefont{X2}{\ttdefault}{cmtt}
\fi


\usepackage[Sonny]{fncychap}
\ChNameVar{\Large\normalfont\sffamily}
\ChTitleVar{\Large\normalfont\sffamily}
\usepackage{sphinx}

\fvset{fontsize=\small}
\usepackage{geometry}

% Include hyperref last.
\usepackage{hyperref}
% Fix anchor placement for figures with captions.
\usepackage{hypcap}% it must be loaded after hyperref.
% Set up styles of URL: it should be placed after hyperref.
\urlstyle{same}

\usepackage{sphinxmessages}
\setcounter{tocdepth}{3}
\setcounter{secnumdepth}{3}


\title{Diacamma Asso}
\date{juin 25, 2019}
\release{2.3.11}
\author{sd-libre}
\newcommand{\sphinxlogo}{\sphinxincludegraphics{DiacammaAsso.jpg}\par}
\renewcommand{\releasename}{Version}
\makeindex
\begin{document}

\ifdefined\shorthandoff
  \ifnum\catcode`\=\string=\active\shorthandoff{=}\fi
  \ifnum\catcode`\"=\active\shorthandoff{"}\fi
\fi

\pagestyle{empty}
\sphinxmaketitle
\pagestyle{plain}
\sphinxtableofcontents
\pagestyle{normal}
\phantomsection\label{\detokenize{index::doc}}



\chapter{Diacamma Asso}
\label{\detokenize{asso/index:diacamma-asso}}\label{\detokenize{asso/index::doc}}
Présentation du logiciel Diacamma Asso.


\section{Présentation}
\label{\detokenize{asso/presentation:presentation}}\label{\detokenize{asso/presentation::doc}}

\subsection{Description}
\label{\detokenize{asso/presentation:description}}
\sphinxstyleemphasis{Diacamma Asso} est un logiciel de gestion spécialement conçu pour les associations sportives ou culturelles.
Avec \sphinxstyleemphasis{Diacamma Asso}, donnez à votre association le logiciel qu’elle mérite! Pas besoin d’être informaticien pour avoir les outils adaptés à votre cas.

L’application de base est entièrement gratuite et vous permet de gérer les accès à vos données et au carnet d’adresses de votre association. Les modules complémentaires vous permettront d’adapter en quelques clics le logiciel à vos besoins.

Les différents modules disponibles vous permettront, par exemple, de:
\begin{itemize}
\item {} 
Gérer plus spécifiquement vos adhérents avec leurs licences sportives et le suivi de leur parcours.

\item {} 
Gérer vos documents de façon centralisée grâce à la gestion documentaire.

\item {} 
Gérer la comptabilité de votre association.

\item {} 
Gérer vos devis, factures et factures proformat pour les adhérents subventionnés par un CE, entre autre.

\end{itemize}

Ce manuel vous aidera dans l’utilisation de ce logiciel.
Si malgré tout, vous ne trouvez pas la réponse à vos problèmes, visiter notre site \sphinxurl{http://www.diacamma.org} où vous trouverez des tutoriels et des astuces.


\subsection{Installation}
\label{\detokenize{asso/presentation:installation}}
Vous pouvez installer \sphinxstyleemphasis{Diacamma Asso} sur un ordinateur dédié à votre association que ce soit un Apple Macintosh (OS X 10.8 et +) ou bien un PC sous MS-Windows (7 et +) ou sous GNU Linux (Ubuntu 14.04 ou +).

\sphinxstyleemphasis{Diacamma Asso} est un logiciel client/serveur : vous pouvez l’installer sur un ordinateur centralisateur et accéder aux données depuis un autre PC connecté au premier, sans limite du nombre d’utilisateurs simultanés.
Si le PC contenant les données est connecté de manière permanente à internet, vous aurez accès à vos données depuis n’importe où dans le monde!

Cette organisation est particulièrement intéressante pour permette à plusieurs cadres associatifs d’avoir accès à des données communes.

Quel responsable ne s’est pas arraché les cheveux suite à un échanges de documents via une clef USB où certaines modifications importantes se perdent?

Pour plus d’information, visiter notre site \sphinxurl{http://www.diacamma.org}


\section{Prise en main}
\label{\detokenize{asso/first_step:prise-en-main}}\label{\detokenize{asso/first_step::doc}}
Le logiciel \sphinxstyleemphasis{Diacamma Asso} comprends un grand nombre de paramétrages et peu paraître difficile à configurer au besoin de votre structure.

Nous vous proposons cette explication pour vous aider à franchir cette première étape dans l’utilisation de cet outil.

Suivez pas à pas les différents phases de réglages. Dans chaque étape, nous ne ré-détaillons pas les fonctionnalités.
Nous vous invitons également à vous référer au reste du manuel utilisateur pour cela.

Il peut être intéressant de réaliser des sauvegardes au cours de cette procédure.
Cela vous permettra, si vous faite une erreur, de revenir à une étape précédente sans tout recommencer (installation comprise).


\subsection{Vérifiez la mise à jours de votre logiciel}
\label{\detokenize{asso/first_step:verifiez-la-mise-a-jours-de-votre-logiciel}}
Commencez par vérifiez que votre logiciel est à jours.
En effet, nous diffusons régulièrement des correctifs qui ne sont pas toujours inclus dans les installateurs.


\subsection{Présentation de votre structures}
\label{\detokenize{asso/first_step:presentation-de-votre-structures}}\begin{quote}

Menu \sphinxstyleemphasis{Général/Nos coordonnées}
\end{quote}

Dans cette écran, vous pouvez décrire les coordonnées de votre structure.
De nombreuses fonctionnalités utilisent ces informations en particulier pour les impressions (facture, comptabilité, listing d’adhérents…)


\subsection{Gestion des adhérents : la saison et les cotisations}
\label{\detokenize{asso/first_step:gestion-des-adherents-la-saison-et-les-cotisations}}\begin{quote}

Menu \sphinxstyleemphasis{Administration/Modules (conf.)/Les saisons et les cotisations}
\end{quote}

Verifiez que vous êtes sur la bonne saison active. Vous pouvez également changer les périodes associées à cette saison et lui associée des documents demandés lié à toutes nouvelles cotisations.
En début de saison, vous pouvez modifier la liste des différentes cotisations que vous proposez à vos adhérents.
Si vous utilisez le module de facturation, l’outil vous permettra d’associé à une cotisation des articles et un montant qui sera utilisé en automatique.


\subsection{Gestion des adhérents : activités et équipes}
\label{\detokenize{asso/first_step:gestion-des-adherents-activites-et-equipes}}\begin{quote}

Menu \sphinxstyleemphasis{Administration/Modules (conf.)/Catégories}
\end{quote}

Dans le cas où vous voulez gérer plusieurs activités, équipes ou trié vos adhérents par classe d’âges, veuillez configurer ces différentes listes.


\subsection{Réglage de votre facturier}
\label{\detokenize{asso/first_step:reglage-de-votre-facturier}}\begin{quote}
\begin{quote}

Menu \sphinxstyleemphasis{Administration/Modules (conf.)/Configuration du règlement}

Menu \sphinxstyleemphasis{Administration/Modules (conf.)/Configuration commercial du facturier}
\end{quote}

Menu \sphinxstyleemphasis{Administration/Modules (conf.)/Configuration financière du facturier}
\end{quote}

Vérifiez que les paramètres du facturier vous correspondent.
Vous pouvez aussi ici, configurer le RIB de votre compte bancaire principal ainsi qu’en ajouter des secondaires.
Ces derniers vous seront utiles pour la fonctionnalité « dépôt de chèques » du facturier.


\subsection{Création de votre premier exercice comptable}
\label{\detokenize{asso/first_step:creation-de-votre-premier-exercice-comptable}}\begin{quote}

Menu \sphinxstyleemphasis{Administration/Modules (conf.)/Configuration comptables}

Menu \sphinxstyleemphasis{Finance/Comptabilité/plan comptable}
\end{quote}

Ouvrer votre premier exercice comptable et rendez le actifs.
Vous devrez aussi créer le plan comptable de cette exercice pour avoir une comptabilité pleinement opérationnel.


\subsection{Ajout des adhérents de l’association}
\label{\detokenize{asso/first_step:ajout-des-adherents-de-l-association}}\begin{quote}

Menu \sphinxstyleemphasis{Association/Adhérents/Adhérents cotisants}

Menu \sphinxstyleemphasis{Administration/Modules (conf.)/Importation de contacts}
\end{quote}

Il vous faut maintenant ajouter les différents adhérents de votre association.
Deux solutions pour cela: un par un ou via une importation extérieur.
Pour ajouter un par un, vous devez ajouter une fiche pour chacun de vos adhérents.
Si vous avez une liste d’adhérents extérieurs à importer, extrayez la sous le format CSV suivant le formalisme expliqué dans le manuel.

N’oubliez pas de leur attribuer une cotisation, sinon ils n’apparaîtrons pas dans la liste des adhérents cotisants.
En cas d’erreur, vous pouvez facilement les retrouver à l’aide de l’outil de recherche.

Notez que pour chaque nouvelle cotisation saisie, une facture non-validée sera générée automatiquement.


\subsection{Mise à jours comptable}
\label{\detokenize{asso/first_step:mise-a-jours-comptable}}\begin{quote}

Menu \sphinxstyleemphasis{Finance/Comptabilité/écritures comptable}

Menu \sphinxstyleemphasis{Finance/Comptabilité/Modèle d’écriture}
\end{quote}

Si vous mettez en place \sphinxstyleemphasis{Diacamma Asso} au cours de votre exercice, vous devrez également saisir votre report à nouveau de l’exercice précédent ainsi que ressaisir les écritures du début d’année.
Attention: n’oubliez pas que l’ajout d’une cotisation d’adhérent génère une facture ainsi que des écritures comptables associées. Prenez en compte dans la reprise de votre comptabilité.
Pour vous aidez dans la saisie de votre comptabilité, nous vous conseillons d’utiliser les modèles d’écritures. Enregistrez entant que modèle les écritures récurrentes que vous avez au cours d’une année. Ainsi vous pouvez rapidement compléter votre comptabilité en quelques cliques.


\subsection{Le courriel}
\label{\detokenize{asso/first_step:le-courriel}}\begin{quote}

Menu \sphinxstyleemphasis{Administration/Modules (conf.)/Paramètrages de courriel}
\end{quote}

Définissez vos réglages pour votre courriel.
Le serveur smpt permettra à \sphinxstyleemphasis{Diacamma Asso} d’envoyé un certain nombre de message: facture en PDF, mot de passe de connexion, …
Vous pouvez préciser comment réagis les liens “écrire à tous” réagis avec votre logiciel de messagerie.


\subsection{Les cadres associatifs}
\label{\detokenize{asso/first_step:les-cadres-associatifs}}\begin{quote}

Menu \sphinxstyleemphasis{Général/Nos coordonnées}

Menu \sphinxstyleemphasis{Administration/Modules (conf.)/Configuration des contacts}

Menu \sphinxstyleemphasis{Association/Adhérents/Adhérents cotisants}
\end{quote}

Dans la fenêtre de vos coordonnées, vous pouvez associé des adhérents comme cadre de votre structure.
Utilisez l’outil de recherche et assignez leur une fonction.
Vous pouvez également rajouter des fonctions propres à votre structure.

Depuis la fiche de chacun de vos adhérents, vous pouvez donner à des personnes actives un droit de connexion à Diacamma Asso.
Privilégié une utilisation du logiciel avec un alias et un mot de passe propre à chaque utilisateur. Associez leur également des droits correspondants à leurs fonctions au sein de votre structure.
Enfin, évitez autant que possible l’utilisation de l’alias “admin” qui doit être réservé pour des actions de configuration et de maintenance.


\subsection{La gestions documentaire}
\label{\detokenize{asso/first_step:la-gestions-documentaire}}\begin{quote}

Menu \sphinxstyleemphasis{Administration/Modules (conf.)/Dossier}

Menu \sphinxstyleemphasis{Bureautique/Documents/Documents}
\end{quote}

Définissez vos différents dossier vous permettant d’importer vos documents à classer et à partager.

Une fois avoir parcouru ces points, votre logiciel \sphinxstyleemphasis{Diacamma Asso} est pleinement opérationnel.
N’hésitez pas à consulter le forum: de nombreuses astuces peux vous aider pour utiliser au mieux votre logiciel.


\chapter{Diacamma adhérent}
\label{\detokenize{member/index:diacamma-adherent}}\label{\detokenize{member/index::doc}}
Aide relative aux fonctionnalités de gestion d’adhésion.


\section{Configuration}
\label{\detokenize{member/config:configuration}}\label{\detokenize{member/config::doc}}
Pour vous permettre d’exploiter au mieux le logiciel, certains paramétrages sont nécessaires.


\subsection{Saisons et cotisations}
\label{\detokenize{member/config:saisons-et-cotisations}}
Le menu \sphinxstyleemphasis{administration/Modules (conf.)/Les saisons et les cotisations} vous permet de modifier à la gestion des licences sportives, en l’occurrence, les saisons sportives et les types de cotisations.

\sphinxstylestrong{Les Saisons}
\begin{quote}

\noindent\sphinxincludegraphics{{season_list}.png}
\end{quote}

Ici, vous pourrez ajouter de nouvelles saisons au fur et à mesure de l’utilisation du logiciel. Vous pourrez également déterminer laquelle est la saison courante (dite active).
De plus, chaque saison est découpée en quatre périodes et douze mois.

C’est la plus petite date de début et la plus grande de fin qui défini la plage de votre saison. Vous pouvez définir des saisons de plus d’un an avec un chevauchement l’une par rapport à l’autre.
Vous pouvez modifier les dates de début et de fin de chaque période. Vous pouvez ajouter ou supprimer une période, mais vous devez toujours en avoir au moins deux.
Deux périodes peuvent se chevaucher ou être disjointes (dans ce cas, un message d’avertissement vous préviens).
Le premier des douze mois commence le mois de la plus petite date de début de période. Même si votre saison couvre plus qu’une année calendaire, il n’y aura pas de treizième mois dans votre saison.

Vous pouvez associer à chaque saison une liste de documents que chaque adhérent devra vous fournir pour finaliser son inscription.
\begin{quote}

\noindent\sphinxincludegraphics{{documents}.png}
\end{quote}

\sphinxstylestrong{Les types de cotisations}
\begin{quote}

\noindent\sphinxincludegraphics{{cotisations}.png}
\end{quote}

Ici vous pourrez saisir les différents types de cotisations proposés par votre association. Par exemple, pour une association pratiquant plusieurs activités sportives distinctes, vous pouvez avoir un type de cotisation pour chaque activité, un autre pour plusieurs de ces activités, et encore des types différents selon une pratique en compétition ou hors compétition des activités.

Quatre modes de durées différentes peuvent être affectées à un type de cotisation:
\begin{itemize}
\item {} 
Annuelle : cotisation couvrant l’ensemble de la saison.

\item {} 
Périodique : cotisation couvrant une période (4 par défaut) de la saison.

\item {} 
Mensuelle : cotisation couvrant un des douze mois de la saison.

\item {} 
Calendaire : cotisation couvrant une année calendaire. Cette cotisation peut donc être à cheval sur deux saisons.

\end{itemize}

Pour le liens avec le module facturier vous devez définir un prix de vente de votre cotisation en créant et associant des articles.
Vous pouvez mettre plusieurs article par cotisation, cela vous permet de distinguer, par exemple, la part de cotisation relative à votre club de la licence de votre fédération.
Avec ces liens entre les cotisations et les articles,vous pourrez générer automatiquement des factures lors de vos procédures d’adhésions. Si une cotisation n’est lié au aucun article, aucune facture ne sera émise.

De même, vous pouvez aussi personnaliser le code comptable du tiers associé à vos adhérants dans le cas d’un création automatique.

\sphinxstylestrong{Les prestation}

Cette option est activable si vous utilisez les catégories d’équipes/cours.

Cela permet d’associer une équipe/cours à un article facturable afin de proposer un choix de prestations supplémentaires moments de la saisi de la cotisation.

Vous choisissez vos prestations au moments de votre prise de cotisation.
Automatiquement, votre adhérent est alors associé dans la bonne catégorie d’équipes/cours défini par la prestation.
De plus, dans la facture d’adhésion est ajouté alors l’article relatif à cet prestation.


\subsection{Catégories}
\label{\detokenize{member/config:categories}}
Le menu \sphinxstyleemphasis{administration/Modules (conf.)/Catégories} vous permet de modifier ce qui peut catégoriser un adhérent: les catégories d’âge, les équipes ou cours et les activités ainsi que la possibilité d’activer ou non ces différents classifications.

Vous pouvez ne pas vouloir utiliser certaines catégories. Pour cela, désactivez là depuis l’écran de paramétrages.
De la même façon, vous pouvez préciser si vous souhaitez pouvoir créer automatiquement une connexion par adhérent actif, afficher un numéro d’adhérent ou gérer des numéros de licence.
Vous pouvez également personnaliser la désignation “équipe” et “activité”.
\begin{quote}

\noindent\sphinxincludegraphics{{categories}.png}
\end{quote}

\sphinxstylestrong{Les âges}

Vous pourrez ici renseigner les catégories d’âges existantes dans votre association en renseignant un nom de catégorie, une année (de naissance) de début et de fin de la catégorie.
\begin{description}
\item[{Vous n’aurez pas besoin de changer les valeurs des années de naissance ultérieurement}] \leavevmode{[}le décalage est effectué automatiquement d’années en années !{]}
\noindent\sphinxincludegraphics{{age}.png}

\end{description}

\sphinxstylestrong{Les équipes/cours}
\begin{description}
\item[{Vous gérez différentes équipes ou différents cours et vous souhaitez pouvoir vos adhérents selon ce critère? Renseignez les ici, vous pourrez alors affecter des adhérents à ces équipes ou cours et ainsi les retrouver plus facilement.}] \leavevmode
\noindent\sphinxincludegraphics{{team}.png}

\end{description}

\sphinxstylestrong{Les activités}

Vous gérez différentes activités (par exemple plusieurs arts martiaux) dans votre association? Les renseigner ici vous permettra ensuite de classer vos adhérents en fonction de ces différentes activités, mais aussi de saisir pour eux plusieurs licences par an si nécessaire.

Exemple : une association regroupant judo et karaté, et donc affiliée à deux fédérations sportives différentes.
Vous pourriez alors saisir 2 licences par adhérent (sous réserve que vos adhérents pratiquent les deux sports et soient licenciés des deux fédérations).
\begin{quote}

\noindent\sphinxincludegraphics{{activity}.png}
\end{quote}


\section{L’adhérent}
\label{\detokenize{member/member:ladherent}}\label{\detokenize{member/member::doc}}
L’adhérent est une personne physique qui a déjà pris une inscription dans votre association.

Pour consulter la liste des adhérents, allez dans le menu \sphinxstyleemphasis{Association/Adhérents/Adhérents cotisants}

\noindent\sphinxincludegraphics{{members}.png}

De cette page vous pouvez imprimer des listes, des étiquettes ou des cartes d’adhérents correspondant au filtrage de cette sélection.
Sélectionner et éditer une ligne pour consulter la fiche de l’adhérent.

\sphinxstyleemphasis{Diacamma Asso} sauvegarde plusieurs types d’informations relative à un adhérent. Vous pouvez les consulter dans la fiche de l’adhérent.

\noindent\sphinxincludegraphics{{file}.png}
\begin{description}
\item[{Cette fiche comporte :}] \leavevmode\begin{itemize}
\item {} 
L’identité de l’adhérent : nom, prénom, adresse, téléphone, date et lieu de naissance…

\item {} 
La liste de ses adhésions.

\item {} 
D’autres onglets dépendants des extensions installées, tels la liste de ses diplômes/formations/grades ou la situation comptable.

\end{itemize}

\end{description}

Si vous avez défini une liste de documents d’inscription dans la saison, vous pouvez, de manière individuelle, cocher les documents rendus. Une impression de l’état de l’ensemble des adhérents en la matière est disponible depuis la liste des adhérents.
Etant donné qu’un adhérent est aussi une personne physique, vous retrouvez ces même fiches depuis la liste des contacts. Il vous est également possible de promouvoir un simple contact physique en adhérent; vous aurez alors juste complété sa fiche.
Enfin, depuis la fiche d’un adhérent, il est possible de lui donner un droit d’accès, ou alias, à Diacamma. (voir Les utilisateurs)

En fonction de vos droits de connexion, il vous sera possible de modifier l’identité d’un adhérent, d’ajouter une adhésion passée ou un diplôme obtenu.

\noindent\sphinxincludegraphics{{modify}.png}

\sphinxstylestrong{Attention:} La loi Informatique et liberté n’autorise la conservation de données que si elle est relative à l’activité de nos adhérents. Veillez donc à bien adapter vos commentaires à ce cadre législatif et informer toute personne de son inscription dans votre base de données.
Consultez le texte de loi en cas de doute, et n’hésitez pas à contacter la CNIL en cas de besoin.


\section{Renouvellements}
\label{\detokenize{member/renew:renouvellements}}\label{\detokenize{member/renew::doc}}
Quand vous commencez une nouvelle saison, vous avez un certain nombre d’adhérents qui renouvellent leur cotisation. Si vous avez déjà enregistré cette personne dans votre logiciel, vous pouvez rechercher sa fiche et lui rajouter une licence pour la nouvelle saison.
Si vous avez beaucoup de renouvellement ce traitement serait très long !

Vous pouvez alors utiliser la liste Adhérents à renouveler
Dans cette liste, vous retrouvez tous les adhérents de la saison précédente qui n’ont pas encore été renouvelés.
Pour leur ajouter la nouvelle cotisation, deux solutions:
\begin{itemize}
\item {} \begin{description}
\item[{De manière unitaire :}] \leavevmode
ouvrez chaque fiche et ajoutez la nouvelle licence.

\end{description}

\item {} \begin{description}
\item[{De manière globale :}] \leavevmode
sélectionnez plusieurs fiches dans la liste et cliquez sur Renouveler. Chaque personne sera alors renouvelée dans les mêmes conditions que la saison précédente (même cotisation, même N° de licence, même équipe, même activité).

\end{description}

\end{itemize}


\section{Rechercher un adhérent}
\label{\detokenize{member/member_search:rechercher-un-adherent}}\label{\detokenize{member/member_search::doc}}
L’outil de recherche des adhérents se situe dans le menu \sphinxstyleemphasis{Association/Adhérents/Recherche d’adhérents}.

\noindent\sphinxincludegraphics{{search}.png}

Vous pouvez alors faire une recherche suivant des critères variés portant sur l’identité ou les adhésions.
Vous pouvez également rechercher des adhérents suivant un des critères des “documents demandés” ou d’un champ personnalisé.

\sphinxstylestrong{Remarque:} si vous faites une recherche avec des critères lié à un cotisation (équipe, documents, numéro licence, …) n’oubliez pas de préciser la saison de recherche.
Le résultat donne une liste d’adhérents correspondants aux critères fournis.

Il est possible de fusionner plusieurs fiches d’une même personne en une seule.
Pour cela vous devez préciser la personne principale, l’outil supprimera les autres fiches après avoir déplacé toutes leurs références sur l’enregistrement principal.
Si vous voulez supprimer un adhérent, celui-ci ne devra pas avoir eu d’activité.

Depuis l’outil de recherche, vous pouvez aussi rechercher les fiches adhérents doublons. Cela vous permet d’afficher la liste des adhérents ayant nom et prénom similaire.
De là, vous pouvez également fusionner ou supprimer la ou les fiches redondantes.


\section{Statistiques des adhérents}
\label{\detokenize{member/statistic:statistiques-des-adherents}}\label{\detokenize{member/statistic::doc}}
\noindent\sphinxincludegraphics{{statistiques}.png}

Vous avez la possibilité d’éditer les statistiques d’adhésion d’une saison donnée.

Ce document vous donne les effectifs par villes et par genres. Il est possible aussi de distinguer les différentes durées de cotisations.


\chapter{Diacamma évenement}
\label{\detokenize{event/index:diacamma-evenement}}\label{\detokenize{event/index::doc}}
Aide relative aux fonctionnalités de gestion d’évenement.


\section{Création d’un évenement}
\label{\detokenize{event/newevent:creation-d-un-evenement}}\label{\detokenize{event/newevent::doc}}
Un évenement, pour \sphinxstyleemphasis{Diacamma}, correspond à la gestion d’une activité particulière de votre structure.
\begin{description}
\item[{Elle peut être de deux types:}] \leavevmode\begin{itemize}
\item {} 
Un passage d’examen
Pour vous aider à gérer un examen au sein de votre association afin de valider pour vos pratiquants un grade, un diplôme ou un niveau.
Il est nécessaire pour cela de configurer la liste des grades/niveaux relatifs à votre structure.

\item {} 
Un stage ou une sortie
Ce type d’évenement va correspond à un grand nombre de cas d’activité dans lequel vous souhaitez gérer une inscription.

\end{itemize}

\end{description}

La création d’un évenement est assez simple.
Commencez simplement à préciser le type, la description et les dates de celui-ci.

Assignez alors à cet évenement un équipe d’organisation (dont un responsable) ainsi que des participants.
Les participants peuvent aussi bien être adhérent ou de simple contact: une colonne permet rapidement de connaitre leur statut.

A chaque participant, vous pouvez assigner un article de facture (un article par défaut simplifie également cette gestion).
Une fois l’évenement validé, une facture est ainsi générée en lien avec l’article du participant.

Dans le cas d’un passage d’examen, à la validation, vous devez également préciser le résultat de chacun des participants.
Tout les participants étant un adhérent de la structure auront une modification de leur fiche afin de faire figurer, le cas échant, leur nouveau grade, diplôme ou niveau.


\section{Configurations}
\label{\detokenize{event/config:configurations}}\label{\detokenize{event/config::doc}}
La configuration du gestionnaire d’évenement est seulement relative à la configuration des grades, diplômes ou niveaux.

Dans le menu \sphinxstyleemphasis{Administration/Configuration des diplômes} vous pouvez préciser la dénomination de vos diplômes
ainsi que saisir hierachiquement les diplômes et éventuellement les sous-diplômes relatifs à votre structures.


\section{Statistique de formation}
\label{\detokenize{event/statistic:statistique-de-formation}}\label{\detokenize{event/statistic::doc}}
Un outil statistique simple vous permet de d’avoir rapidement un résumé pour une saison donnée du nombre de diplômes attribué pour cette periode de temps.


\chapter{Diacamma comptabilité}
\label{\detokenize{accounting/index:diacamma-comptabilite}}\label{\detokenize{accounting/index::doc}}
Aide relative aux fonctionnalités comptables.


\section{Definitions}
\label{\detokenize{accounting/definition:definitions}}\label{\detokenize{accounting/definition::doc}}\begin{quote}

\sphinxstylestrong{Remarques:} Ce module comptable est proche d’une comptabilité type « entreprise », néanmoins elle ne respecte pas certaines exigences légales et fiscale en la matiére.
Ce modules ne peux pas étre utilisé pour la tenu de compte de structures commerciales, concurrentielles ou professionnelles mais seulement des structures de type associative gérées par des bénévoles.
Le représentant légale de la structure utilisant ce module doit s’assurer que sa comptabilité respecte alors la législation de son pays en vigueur.
\end{quote}


\subsection{Exercice comptable}
\label{\detokenize{accounting/definition:exercice-comptable}}
Un exercice comptable est une période de temps sur laquelle une
personne morale (entreprise, association ou autre) enregistre tous les
mouvements d’argent la concernant.

Cette période est généralement de 12 mois consécutifs du 1er janvier au 31 décembre mais peut varier
d’une entité à une autre. La durée légale est toutefois fixée à un
maximum de 2 ans. La durée de l’exercice est fixée à l’avance et ne
peut être modifiée que sur décision du conseil d’administration.


\subsection{Tiers comptable}
\label{\detokenize{accounting/definition:tiers-comptable}}
Un tiers comptable est une personne physique ou morale avec
laquelle une entité va avoir des échanges monétaires (clients,
fournisseurs, salariés, administrations…).


\subsection{Journaux comptables}
\label{\detokenize{accounting/definition:journaux-comptables}}
Un journal comptable est un regroupement d’écritures comptables permettant de classer celles-ci.

Les journaux par défaut sont:
* journal d’achat contenant toutes les écritures relatives aux achats fait par une entité
* journal de vente contenant toutes les écritures relatives aux dépenses effectuées par une entité
* journal des encaissements contenant toutes les écritures relatives aux mouvement sur les comptes en monétaire (compte bancaires, compte caisse…) en relation avec les dépenses et recettes de l’entité
* journal des reports à nouveau contenant les écritures permettant le passage d’un exercice comptable à son suivant
* journal des opérations diverses contenant l’ensemble des autres écritures (ex: frais financiers…)


\subsection{Ecritures comptables}
\label{\detokenize{accounting/definition:ecritures-comptables}}
Une écriture comptable est un ensemble de lignes inscrites dans divers
comptes comptables permettant un équilibre.
La somme des crédits d’une écriture doit donc être égale à la somme des débits de cette même écriture.

Par exemple, une écriture d’achat se schématise par:
* une ligne au crédit du compte tiers fournisseur représentant l’ensemble de la somme de la facture
* une ou plusieurs lignes au débit des comptes de charges correspondants au type de ressources achetées (matériel, service…)

Le total des lignes dans les comptes de charge est donc égal au montant
porté sur la ligne de compte tiers fournisseur.

Les écritures comptables d’encaissement peuvent et doivent être pointées afin de marquer le rapprochement avec les comptes bancaires et
la caisse physique. De cette façon, on peut suivre facilement les écritures passées dans la comptabilité mais non encore effectives dans
la réalité. Le pointage est nécessaire pour le passage d’un exercice comptable à son suivant.

Il est également possible et recommandé de lettrer les écritures, c’est à dire de créer un sous ensemble
cohérent d’écritures en provenance de journaux divers afin de stipuler qu’elle correspondent à la même opération de la vie réelle.

Ex: l’écriture comptable d’achat d’un bien peut être lettrée avec son écriture d’encaissement.


\subsection{Plan comptable de l’exercice}
\label{\detokenize{accounting/definition:plan-comptable-de-l-exercice}}
Le plan comptable de l’exercice est l’ensemble des comptes utilisés au
cours d’un exercice comptable en se basant sur le plan comptable
couramment admis par l’administration fiscale.

les numéros de comptes doivent impérativement commencer par le préfixe donné par le
plan comptable en vigueur au moment de la création du compte.


\section{Exercice}
\label{\detokenize{accounting/fiscalyear:exercice}}\label{\detokenize{accounting/fiscalyear::doc}}

\subsection{Paramétrages}
\label{\detokenize{accounting/fiscalyear:parametrages}}\begin{quote}

\noindent\sphinxincludegraphics{{parameters}.png}
\end{quote}

Initialement, vous pouvez définir ici le type de système comptable que
vous voulez utiliser (ex: Plan comptable générale Français).
\sphinxstyleemphasis{Attention:} une fois défini, ce système n’est plus modifiable.

Vous pouvez changer également la monnaie courrante de votre comptabilité.


\subsection{Création d’un exercice comptable}
\label{\detokenize{accounting/fiscalyear:creation-d-un-exercice-comptable}}
Pour créer un exercice comptable, rendez vous dans le menu \sphinxstyleemphasis{Administration/Modules (conf.)/Configuration comptable}.
\begin{quote}

\noindent\sphinxincludegraphics{{fiscalyear_list}.png}
\end{quote}

De là, cliquez sur Ajouter afin de faire apparaître le formulaire vous permettant de renseigner les bornes de l’exercice
\begin{quote}

\noindent\sphinxincludegraphics{{fiscalyear_create}.png}
\end{quote}

Indiquez la date de début (celle-ci doit étre le lendemain de la date
de clôture de l’exercice précédent) et la date de fin (au maximum 2 ans
aprés le début de l’exercice) de l’exercice puis cliquez sur le bouton
OK.

Votre nouvel exercice sera alors disponible dans la
liste des exercices. Pour en continuer la création, il vous faudra le
sélectionner dans la liste et cliquer sur le bouton Activer afin de
pouvoir travailler dessus par défaut.

Depuis ce même écran de configuration, vous pouvez également modifier
ou ajouter des journaux.
Vous pouvez également créer des champs personnalisés (comme pour la fiche de contacte)
pour la fiche de tiers. Ceci peut être interessant si vous voulez réaliser des recherches/filtrages
sur des informations propres à votre fonctionnement.

Fermez maintenant la liste des exercices afin de vous rendre dans la comptabilité et
pouvoir créer le plan comptable de l’exercice ainsi qu’affecter le
report à nouveau avant de pouvoir commencer éà saisir des écritures.

Pour ce faire, rendrez-vous dans le menu \sphinxstyleemphasis{Financier/Comptabilité/plan   comptable}
\begin{quote}

\noindent\sphinxincludegraphics{{account_list}.png}
\end{quote}

Ici, commencez par créer les comptes de base de votre exercice.

Si vous avez déjà un précédent exercice, vous pouvez en importer la liste de code comptable.
\begin{description}
\item[{Une fois ceci fait, plusieurs choix se présentent à vous:}] \leavevmode\begin{itemize}
\item {} \begin{description}
\item[{Il s’agit de votre premier exercice comptable}] \leavevmode
Vous venez de créer votre structure, vous n’avez pas de report à nouveau, cliquez donc dès maintenant sur le bouton commencer, vous aurez alors achevé la création de votre exercice.

\end{description}

\item {} \begin{description}
\item[{Il ne s’agit pas de votre premier exercice mais vous n’utilisiez pas ce logiciel avant}] \leavevmode
Il va falloir saisir manuellement votre report à nouveau.
Pour cela, sortez du plan comptable de l’exercice, entrez dans la liste des écritures et saisissez manuellement une écriture complète, cohérente et équilibrée pour votre report à nouveau.
Une fois ceci fait, retournez dans votre plan comptable pour cliquer sur commencer.

\end{description}

\item {} \begin{description}
\item[{Il ne s’agit pas de votre premier exercice et vous utilisiez déjà ce logiciel}] \leavevmode
Utilisez le bouton « report à nouveau » » afin d’importer le résultat de l’exercice précédent.
Comme il n’est pas possible de commencer un exercice avec un résultat (qu’il soit bénéficiaire ou déficitaire).
Vous devez avant de commencer votre exercice, ventiler cette somme sur un compte de capitaux (capital, réserve, ..).
La décision de cette affectation est prise par le conseil d’administration sous le contrôle de votre vérificateur aux comptes.
Pour cela, vous pouvez créer une écriture spécifique (journal “report à nouveau”) ou utilisez le questionnaire l’or du commencement de l’exercice.
Pour commencer l’exercice, clique sur le bouton afin de clore cette phase de création.

\end{description}

\end{itemize}

\end{description}


\subsection{Création, modification et édition de comptes dans le plan}
\label{\detokenize{accounting/fiscalyear:creation-modification-et-edition-de-comptes-dans-le-plan}}
Plaçons nous dans le menu \sphinxstyleemphasis{finance/comptabilité/plan comptable de l’exercice}.

A tout moment au cours d’un exercice vous pouvez être amener à ajouter un nouveau compte dans votre plan.
\begin{quote}

\noindent\sphinxincludegraphics{{account_new}.png}
\end{quote}

Référez vous aux codes légaux définis par la réglementation de votre pays pour définir correctement les 3 premiers chiffres.
Pour les associations dépendant du droit français, vous pourrez trouver des informations sur le site du gouvernement des finances français (\sphinxurl{http://www.minefe.gouv.fr/themes/entreprises/compta\_entreprises/index.htm}).
Les 3 derniers chiffres du compte vous sont propres suivant votre besoin. Modifiez la désignation pour simplifier l’identification de votre compte.

Si vous vous étes trompé, vous pouvez changer le compte et sa désignation. Si des écritures ont été saisies avec ce compte, elles seront automatiquement migrées.

Par contre, le nouveau compte doit rester dans la même catégorie comptable.
Vous pouvez consulter un compte précis. Vous pouvez alors voir
l’ensemble des lignes d’écritures associées à ce compte, ainsi que la
valeur du compte au début (report à nouveau) et la valeur actuelle.
\begin{quote}

\noindent\sphinxincludegraphics{{account_edit}.png}
\end{quote}

Il vous est aussi possible de supprimer un compte du plan si aucune opération n’y a été réalisée.


\subsection{Clôture d’un exercice}
\label{\detokenize{accounting/fiscalyear:cloture-d-un-exercice}}
A la fin de la période, vous devez clôturer votre exercice. Cette
opération, définitive, se réalise sous le contrôle de votre
vérificateur aux comptes.
Dans le menu \sphinxstyleemphasis{Financier/Comptabilité/plan comptable}, cliquez sur le bouton « Clôturer ».

\sphinxstylestrong{Attention:} Toutes les écritures doivent être validées avant de commencer cette procédure.

La phase de validation va réaliser un traitement consistant à
créer une série d’écritures de fin d’exercice résumant le résultat et
les dettes tiers (factures clients ou fournisseurs transmises mais pas encore réglées).


\section{Tiers comptable}
\label{\detokenize{accounting/third:tiers-comptable}}\label{\detokenize{accounting/third::doc}}

\subsection{Création d’un tiers}
\label{\detokenize{accounting/third:creation-d-un-tiers}}
Plaçons nous dans le menu \sphinxstyleemphasis{Finance/Tiers}.

\noindent\sphinxincludegraphics{{third_list}.png}

La liste des tiers précédemment enregistrés apparaît.
Vous pouvez réaliser un certain nombre de fitrage rapide suivant le nom ou
la situation. Vous pouvez alors imprimer la liste.
Pour ajouter un nouveau tiers, vous devez commencer par choisir un contact (physique
ou moral) associé à ce tiers comptable.

\noindent\sphinxincludegraphics{{third_add}.png}

Depuis cet écran, vous pouvez aussi créer un nouveau contact avant de le sélectionner.

\noindent\sphinxincludegraphics{{third_edit}.png}

Pour chaque tiers, vous pouvez associer des comptes comptables
correspondant à la nature de vos tiers: fournisseur, client, salarié et
sociétaire. Vous pouvez changer ces comptes pour imputer dans votre
comptabilité comme vous le souhaitez cette personne au cours
d’opérations financières.


\subsection{Situation d’un tiers}
\label{\detokenize{accounting/third:situation-d-un-tiers}}
La fiche d’un tiers vous permet d’avoir une vue globale de l’état des recettes et dépenses liées à ce tiers.

\noindent\sphinxincludegraphics{{third_state}.png}

Vous retrouverez ici l’ensemble des écritures comptables de
l’exercice liées à ce tiers. Vous trouverez également un résumé des
débits et crédits permettant en un seul regard de savoir s’il reste des
dettes impayées. Avec d’autres modules financiers, vous pourrez
également consulter des opérations liées.


\subsection{Configuration}
\label{\detokenize{accounting/third:configuration}}
Depuis le menu \sphinxstyleemphasis{Administration/Modules (conf.)/Configuration comptable} vous avez la possibilité d’ajouter à tout tiers des champs personnalisés.
Le mécanisme est similaire à ce que vous pouvez trouver dans la configuration des contacts.


\section{Ecritures}
\label{\detokenize{accounting/entity:ecritures}}\label{\detokenize{accounting/entity::doc}}

\subsection{Saisie d’une écriture}
\label{\detokenize{accounting/entity:saisie-d-une-ecriture}}\begin{quote}

\sphinxstylestrong{Cas général}
\end{quote}

Plaçons nous dans le menu \sphinxstyleemphasis{Financier/Comptabilité/écritures comptables}.
\begin{quote}

\noindent\sphinxincludegraphics{{entity_list}.png}
\end{quote}

Depuis cet écran, nous avons la possibilité de visualiser les écritures
précédemment saisies ainsi que d’en ajouter de nouvelles.

Comme vous pouvez le voir dans cet écran, vous pouvez consulter les écritures
par journaux ou par état. 5 filtres d’état vous sont proposés:
\begin{itemize}
\item {} 
Tout: aucun filtrage n’est appliqué

\item {} 
En cours (Brouillard): seulement les écritures non encore validées

\item {} 
Validé: seulement les écritures déjé validées

\item {} 
Lettré: seulement les écritures rapprochées ou lettrées avecd’autres

\item {} 
Non lettré: seulement les écritures non encore lettrées

\end{itemize}

Ainsi que 5 journaux par défaut:
* Journal des achats
* Journal des ventes
* Journal des encaissements
* Journal des opérations diverses
* Journal de report à nouveaux

Pour ajouter une écriture, commençons d’abord par sélectionner sur quel
journal nous souhaitons réaliser notre nouvelle écriture, puis cliquons
sur le bouton \sphinxstyleemphasis{Ajouter}.

Aprés avoir précisé les dates de votre écriture, il vous faut
ajouter les différentes lignes correspondant à votre opération financiére.

Pour ajouter une ligne d’écriture, saisissez son code comptable
si vous le connaissez dans la zone d’ajout ou laissez-vous guider par
l’assistant en cliquant sur le bouton correspondant au type de compte désiré.
\begin{quote}

\noindent\sphinxincludegraphics{{entity_add}.png}
\end{quote}

L’outil ne vous permettra pas de valider votre écriture si elle est déséquilibrée.

Quand on débute en comptabilité, on a parfois du mal pour savoir si une ligne est en débit ou en crédit. Pour vous aider, un message
vous avertit si vous avez saisi un remboursement ou un avoir et un bouton vous permettez d’inverser trés facilement votre écriture si besoin.
\begin{quote}

\sphinxstylestrong{Réaliser un encaissement}
\end{quote}

Une écriture d’encaissement peut se saisir manuellement comme précédemment mais bien souvent, un réglement vient compléter un achat ou une vente effectué quelques jours plus tét.

Pour simplifier votre saisie, ré-ouvrez l’écriture d’achat ou de vente dont vous souhaitez saisir le réglement, cliquez sur le bouton « Encaissement »: l’application vous propose alors une nouvelle écriture
partiellement remplie. Il ne vous reste plus qu’é préciser sur quel compte financier (caisse, banque…) vous voulez réaliser cette opération.

Une fois un encaissement validé via ce mécanisme, les deux écritures (celle d’achat ou de vente et celle d’encaissement) sont automatiquement lettrées.
\begin{quote}

\sphinxstylestrong{Ecriture de report à nouveau}
\end{quote}

Le journal de report à nouveau n’est modifiable que dans la phase d’initialisation de votre exercice.

A ce moment, vous pouvez étre amené à réaliser des opérations spécifiques comme par exemple la ventilation des bénéfices de l’année
précédente suivant plusieurs comptes.

Par contre, dans ce journal, il n’est pas possible d’ajouter des lignes d’écritures de charges ou de produits.


\subsection{Lettrage d’écritures}
\label{\detokenize{accounting/entity:lettrage-d-ecritures}}
Comme nous l’avons évoqué dans un précédent chapitre, il est régulier
qu’un ensemble d’écritures se référent à une ou plusieurs opérations
communes. Dans ce cas, vous pouvez marquer ces écritures comme étant
liées: vous allez alors les lettrer.

Le plus souvent, le lettrage
se réalise entre écritures de valeur de tiers complémentaires: entre
une écriture d’achat (ou de vente) et son encaissement associé.

Mais, il peut arriver que nous souhaitions lettrer plus de deux
écritures. Par exemple, vous pouvez vouloir régler 3 factures d’un
fournisseur en une seule fois. Dans ce cas, comme vous ne faites qu’un
seul chéque d’un montant égal à la somme des factures, vous n’aurez
qu’une écriture d’encaissement que vous allez lettrer avec les 3
écritures d’achats. A la relecture de votre comptabilité, il deviendra
alors simple de comprendre qu’il s’agissait d’un réglement multiple.

Pour réaliser cette action, sélectionnez les lignes d’écritures que vous désirez
lier et cliquez sur le bouton « Lettrer »: Si l’outil les considére comme
étant cohérentes, il réalisera le lettrage symbolisé par un numéro
commun à ces écritures en derniére colonne du journal.
\begin{description}
\item[{Voilà les règles pour qu’un lettrage soit accepté:}] \leavevmode\begin{itemize}
\item {} 
Les lignes d’écritures doivent être des lignes de tiers (code comptable 4xx).

\item {} 
Les lignes d’écritures doivent appartenir au même exercice.

\item {} 
Les lignes d’écritures doivent être associée au même code de tiers et au même tiers

\item {} 
Les lignes d’écritures doivent d’équilibrée.

\end{itemize}

\end{description}

Si vous cliquez à nouveau sur ce bouton, vous avez la possibilité de supprimer
le lettrage de cette ligne d’écriture ainsi que celui des écritures associées.


\subsection{Validation d’écritures}
\label{\detokenize{accounting/entity:validation-d-ecritures}}
Par défaut, une écriture est saisie au brouillard, c’est à dire dans un
état où elle reste modifiable ou supprimable.

Par contre, il est nécessaire, pour finaliser votre comptabilité, de valider cette
écriture pour entériner votre saisie.

Pour réaliser cette action, sélectionnez les écritures contrôlées et
cliquez sur le bouton « Valider »: L’application affectera alors un
numéro à vos écritures ainsi que la date de validation.

Une fois validée, une écriture devient non modifiable: ce mécanisme assure le
caractére intangible et irréversible de votre comptabilité.
Comme l’erreur est humaine, au lieu de supprimer un écriture valider, il vous faudra
créer une écriture inverse de pour l’annuler.

Cela est utile pour un responsable comptable pour préciser que cette
écriture est vérifiée par rapport au justificatif associé.
Cela sert aussi, dans le cas des écritures d’encaissements, de contrôler que
cette recette ou dépense figure bien sur un relevé de banque.

Pour clôturer un exercice, l’ensemble des écritures doivent étre validées.


\subsection{Recherche d’écriture}
\label{\detokenize{accounting/entity:recherche-d-ecriture}}
Depuis la liste des écritures, le bouton « Recherche » vous permet
de définir des critères de recherche d’écritures comptables.
\begin{quote}

\noindent\sphinxincludegraphics{{entity_search}.png}
\end{quote}

En cliquant sur “Rechercher », l’outil va rechercher dans la base
toutes les écritures correspondantes à ces critères. Vous pourrez alors
imprimer cette liste ou éditer/modifier une écriture.


\section{Comptabilité analytique}
\label{\detokenize{accounting/costaccounting:comptabilite-analytique}}\label{\detokenize{accounting/costaccounting::doc}}
Pour permettre de réaliser une analyse financière des différentes activités de votre structure, vous pouvez mettre en place une comptabilité analytique.

La comptabilité analytique proposée par le logiciel est une version simplifiée.
En effet, il n’est pas possible de ventiler une même ligne écriture sur plusieurs codes analytiques.


\subsection{Les codes analytiques}
\label{\detokenize{accounting/costaccounting:les-codes-analytiques}}
Depuis le menu \sphinxstyleemphasis{Financier/Comptabilité/Comptabilités analytiques}, vous accédez à la liste des codes.

Depuis cet écran vous pouvez créer, modifier ou supprimer un code analytique. Celui-ci est constitué
en plus d’un titre et d’un description, d’un status (Ouvert ou Clôturé).

Depuis cette liste, vous obtiendrez également le résultat comptable (les produits diminués des charges) de votre code analytique.
\begin{quote}

\noindent\sphinxincludegraphics{{costaccount_list}.png}
\end{quote}

Par défaut, un filtrage vous pemet de ne voir que les code analytique courant. Cliquez dans la coche pour désactiver ce filtre.

A noter que dans le menu \sphinxstyleemphasis{Administration/Modules (conf.)/Configuration comptable}, vous avez un paramètre vous permettant
de rendre obligatoire une affectation analytique à toutes charges ou produits.


\subsection{Imputation analytique d’une écriture}
\label{\detokenize{accounting/costaccounting:imputation-analytique-d-une-ecriture}}
Si vous avez des codes analytiques ouverts, vous pouvez imputer une ligne écriture sur l’un d’entre eux.
\begin{quote}

\noindent\sphinxincludegraphics{{costaccount_assign}.png}
\end{quote}

Pour cela, éditez votre écriture (validée ou non) à imputer à votre code analytique, et modifiez la ligne d’écriture avec l’affectation désirée.

Il est aussi possible de réaliser cette imputation par lot depuis la liste des écritures.
Pour cela, sélectionnez les écritures à affecter et cliquez sur \sphinxstyleemphasis{Analytique}: choisissez alors le nouveau code à utiliser
pour l’ensemble des lignes d’écritures de charges ou de produits.


\subsection{Impressions analytiques}
\label{\detokenize{accounting/costaccounting:impressions-analytiques}}
Depuis la liste des codes analytiques, vous pouvez réaliser un rapport type « compte de résultats ».

Ce rapport est les équivalents de ceux obligatoires pour une comptabilité sur un exercice mais adaptés au besoin d’une comptabilité analytique.


\section{Modèle}
\label{\detokenize{accounting/model:modele}}\label{\detokenize{accounting/model::doc}}

\subsection{Déclaration d’un modèle}
\label{\detokenize{accounting/model:declaration-d-un-modele}}
Un modèle d’écriture est un ensemble de lignes d’écritures mémorisées à l’avance que vous pouvez rejouer aussi souvent que vous le voulez.
Depuis le menu Financier/Comptabilité/Modèles d’écritures, vous accédez à la liste des modèles.
\begin{quote}

\noindent\sphinxincludegraphics{{model_list}.png}
\end{quote}

Un modèle est associé à un journal et contient une description ainsi qu’une liste de lignes comprenant un code comptable et un montant.
\begin{quote}

\noindent\sphinxincludegraphics{{model_item}.png}
\end{quote}

Nous vous conseillons de créer un modèle pour chacune de vos dèpenses (ou recette) règulières. Ainsi, vous gagnerez du temps sur la saisie de votre comptabilitè sans avoir à rechercher le bon code comptable.


\subsection{Utilisation d’un modèle}
\label{\detokenize{accounting/model:utilisation-d-un-modele}}
L’utilisation d’un modèle est très simple. Avec le bouton Modèle présent dans les liste d’écritures, une sélection de modèle vous est présentée.
\begin{quote}

\noindent\sphinxincludegraphics{{model_add}.png}
\end{quote}

Sélectionnez votre modèle et précisez un coefficient multiplicateur. Ce facteur est très pratique lorsque l’on a des factures récurrentes mais dont le montant peut fluctuer. Il est alors possible, è l’aide de ce rèel, de l’affiner.
Une fois validé, une écriture est générée suivant la description du modèle. Vous pouvez la corriger comme n’importe quelle écriture.


\section{Budget prévisionnel}
\label{\detokenize{accounting/budget:budget-previsionnel}}\label{\detokenize{accounting/budget::doc}}

\subsection{Budget par analytique}
\label{\detokenize{accounting/budget:budget-par-analytique}}
Depuis l’interface des comptabilités analytiques, vous pouvez ajouter un budget prévisionnel à chacun.
Cliquez simplement sur le bouton \sphinxstyleemphasis{Budget} après avoir sélectionner une comptabilité à compléter.

L’interface vous permet alors d’ajouter des comptes de charges ou de produits ainsi qu’un solde prévisionnel.
vous pouvez également importer les montants des charges et produits du résultat d’une comptabilités précédentes.

Ce budget prévisionnel apparait alors dans les rapports afin de le comparer avec la comptabilité réalisé.


\subsection{Budget par exercice}
\label{\detokenize{accounting/budget:budget-par-exercice}}
Depuis l’interface du plan comptable courant, vous pouvez ajouter un budget prévisionnel à l’exercice via le bouton \sphinxstyleemphasis{Budget}.

Comme pour le budget analytique, vous pouvez ajouter des comptes de charges ou de produits ainsi que d’importer le résultat de l’exercice précédent.
A noter qu’automatiquement, l’ensemble des budgets analytiques associés au même exercice sont automatiques consolidés dans ce budget d’exercice.

Le \sphinxstyleemphasis{resultat d’exercice} presente également la comptabilité courant en affichant également le budget prévisionnel à des fins de comparaison.


\section{Raports}
\label{\detokenize{accounting/reporting:raports}}\label{\detokenize{accounting/reporting::doc}}
Dans le catégorie \sphinxstyleemphasis{Financier/Comptabilité}, vous avez accès à un ensemble de rapports relatifs à votre comptabilité.
Vous pouvez consulter votre rapport ainsi qu’en réaliser une impression via une génération PDF de ce rapport.
De plus, pour tout les exercices non-cloturés, vous pouvez préciser une période de consultation.
Ceci est principalement utile pour vous aider à faire vos rapprochements bancaires ou si vous avez besoins d’éditer des situations financiéres trimestrielles.


\subsection{Compte de résultats}
\label{\detokenize{accounting/reporting:compte-de-resultats}}
Le compte de résultat est un document comptable synthétisant l’ensemble des charges et des produits d’une entreprise ou autre organisme ayant une activité marchande, au cours d’un son exercice comptable.
Ce document donne le résultat net, c’est-à-dire ce que l’entreprise a gagné (bénéfice) ou perdu (perte) au cours de la période.


\subsection{Bilan}
\label{\detokenize{accounting/reporting:bilan}}
Le bilan est une photographie du patrimoine de l’entreprise qui permet de réaliser une évaluation d’entreprise, et plus précisément de savoir après retraitement (par exemple d’une optique patrimoniale à celle sur option de juste valeur pour l’adoption des normes internationales) combien elle vaut et si elle est solvable.
Il existe donc trois finalités au bilan:
\begin{itemize}
\item {} 
Le bilan comptable interne, généralement détaillé, utilisé par les responsables de l’entreprise pour différentes analyses internes;

\item {} 
Le bilan comptable officiel, destiné aux contrôleurs de la comptabilité (auditeurs et commissaires aux comptes) et aux actionnaires (et plus généralement aux tiers);

\item {} 
Le bilan fiscal, qui sert à déterminer le bénéfice imposable;

\end{itemize}


\subsection{Grand livre}
\label{\detokenize{accounting/reporting:grand-livre}}
Le Grand livre est le recueil de l’ensemble des comptes utilisés d’une entreprise qui tient sa comptabilité en partie double.

Des options de filtrage sont à votre disposition:
* L’exercice et la plage de dates désirées
\begin{quote}

Afin de consulter que les opérations de la période concerné.
\end{quote}
\begin{itemize}
\item {} 
Code comptable commençant par

\end{itemize}
\begin{quote}

En indiquant le début de code d’un compte, vous effectuerez un filtrage avec les opérations concernées par seulement ces comptes.
\end{quote}
\begin{itemize}
\item {} 
Seulement les écritures non-lettrées

\end{itemize}
\begin{quote}

En cochant cette coche, vous effectuerez n’afficherez que les opérations n’étant pas rapprocher par lettrage.
Noter que seul le lettrage de ligne d’écriture de tiers n’a de sens.
\end{quote}


\subsection{Balance}
\label{\detokenize{accounting/reporting:balance}}
La balance comptable est un état d’une période, établi à partir de la liste de tous les comptes du grand livre de l’entreprise (qu’ils soient de bilan ou de gestion) et regroupant tous les totaux (ou masses) en débit et crédit de ces comptes et par différence tous les soldes débiteurs et créditeurs.

Des options de filtrage sont à votre disposition:
* L’exercice et la plage de dates désirées
\begin{quote}

Afin de consulter que les opérations de la période concerné.
\end{quote}
\begin{itemize}
\item {} 
Code comptable commençant par

\end{itemize}
\begin{quote}

En indiquant le début de code d’un compte, vous effectuerez un filtrage avec les opérations concernées par seulement ces comptes.
\end{quote}
\begin{itemize}
\item {} 
Seulement les non-soldés

\end{itemize}
\begin{quote}

Permet, en cochant cette coche, de n’afficher que les lignes n’ayant pas un solde nul.
\end{quote}
\begin{itemize}
\item {} 
Détail par tiers

\end{itemize}
\begin{quote}

En cochant cette coche, vous afficherez pour les comptes de tiers le détail de leur balance par tiers.
\end{quote}


\subsection{Listing des écritures}
\label{\detokenize{accounting/reporting:listing-des-ecritures}}
Depuis l’écran de la liste des écritures comptables, vous avez la possibilité d’exporter l’ensemble des écritures de l’exercice.
Vous pourrez visualiser, imprimer, exporter au format PDF ou CSV (permet l’import de vos écritures dans un tableur).


\subsection{Listing du plan comptable de l’exercice}
\label{\detokenize{accounting/reporting:listing-du-plan-comptable-de-l-exercice}}
Depuis l’écran du plan comptable de l’exercice, vous avez la possibilité d’exporter l’ensemble des écritures de code comptable utilisés et leur solde du moment.
Vous pourrez visualiser, imprimer, exporter au format PDF ou CSV (permet l’import de vos écritures dans un tableur).


\chapter{Facturier Diacamma}
\label{\detokenize{invoice/index:facturier-diacamma}}\label{\detokenize{invoice/index::doc}}
Aide relative aux fonctionnalités de gestion de factures.


\section{Les articles}
\label{\detokenize{invoice/articles:les-articles}}\label{\detokenize{invoice/articles::doc}}

\subsection{Création et modification}
\label{\detokenize{invoice/articles:creation-et-modification}}
Depuis le menu \sphinxstyleemphasis{Finance/Facturier/Les articles} vous avez la possibilité de définir l’ensemble de vos articles facturables.

\noindent\sphinxincludegraphics{{articles_list}.png}

Vous pouvez ajouter, modifier ou supprimer un article. La suppression n’est pas possible si l’article est utilisé dans une facture.

A chaque article, vous devez définir un code comptable d’imputation pour la génération d’écritures automatique.

Le champ \sphinxstyleemphasis{stockable} permet de définir si vous voulez gérer une gestion de stock de cet article:
\begin{itemize}
\item {} \begin{description}
\item[{non stockable}] \leavevmode
Article sans gestion de stock, comme par exemple des articles de service.

\end{description}

\item {} \begin{description}
\item[{stockable}] \leavevmode
Article stockable et facturable.

\end{description}

\item {} \begin{description}
\item[{stockable \& non vendable}] \leavevmode
Article stockable non proposable à la vente.
Utile pour suivre des stocks de matériel interne.

\end{description}

\end{itemize}

De plus, dans le cas où vous réalisez des factures avec TVA, vous devrez préciser, pour chaque articles, le taux de taxe à appliquer.

L’onglet \sphinxstyleemphasis{Fournisseur} permet d’identifier des références fournisseurs pour simplifier leur commande ou leur référencement.

Le champ \sphinxstyleemphasis{code d’imputation comptable} permet d’associer à cet article une configuration comptable (voir \sphinxstyleemphasis{Configuration et paramétrage})


\subsection{La facture avec TVA}
\label{\detokenize{invoice/articles:la-facture-avec-tva}}
Si vous êtes soumis à la TVA, l’édition de la facture change un peu

En plus de préciser si les articles sont en montant HT ou TTC, vous avez en bas de l’écran le total de la facture hors-taxe, taxes comprises ainsi que le montant total des taxes.

De plus, dans l’impression de la facture, un sous-détail des taxes apparait par taux de TVA.


\section{Création de facture}
\label{\detokenize{invoice/create_bill:creation-de-facture}}\label{\detokenize{invoice/create_bill::doc}}

\subsection{Création}
\label{\detokenize{invoice/create_bill:creation}}
Depuis le menu \sphinxstyleemphasis{Finance/Facturier/Les factures} vous pouvez éditer ou ajouter une nouvelle facture.

Commencez par définir le type de document (devis, facture, reçu ou avoir) que vous souhaitez créer ainsi que la date d’émission et un commentaire qui figurera dessus.

Dans cette facture, vous devez préciser le client associé, c’est à dire le tiers comptable imputable de l’opération.
\begin{quote}

\noindent\sphinxincludegraphics{{bill_edit}.png}
\end{quote}

Ensuite ajoutez ou enlevez autant d’articles que vous le désirez.
\begin{quote}

\noindent\sphinxincludegraphics{{add_article}.png}
\end{quote}

Par défaut, vous obtenez la désignation et le prix par défaut de l’article sélectionné, mais l’ensemble est modifiable. Vous pouvez choisir aussi l’article divers: aucune information par défaut n’est alors proposé.

Si l’article a été défini comme \sphinxstyleemphasis{stockable}, vous devrez en plus préciser depuis quel lieu de stockage il sera sortie.
Il n’est bien sur pas possible de vendre plus d’article stockable que l’on possède dans le stock.


\subsection{Changement d’état}
\label{\detokenize{invoice/create_bill:changement-d-etat}}
Depuis le menu \sphinxstyleemphasis{Finance/Facturier/Les factures} vous pouvez consulter les factures en cours, validé ou fini.

Un devis, une facture, un reçu ou un avoir dans l’état « en cours » est un document en cours de conception et il n’est pas encore envoyé au client.

Depuis la fiche du document, vous pouvez le valider: il devient alors imprimable et non modifiable.

Dans ces deux cas, une écriture comptable est alors automatiquement générée.

Un devis validé peut facilement être transformé en facture dans le cas de son acceptation par votre client. La facture ainsi créé se retrouve alors dans l’état « en cours » pour vous permettre de la réajuster.

Une fois qu’une facture (ou un avoir) est considéré comme terminée (c’est à dire réglée ou définie comme pertes et profits), vous pouvez définir son état à «fini».

Depuis une facture « fini », il vous est possible de créer un avoir correspondant à l’état « en cours ». Cette fonctionnalité vous sera utile si vous êtes amené à rembourser un client d’un bien ou un service précédemment facturé.

Si une facture contiens des articles \sphinxstyleemphasis{stockable}, un bordereau de sortie est automatiquement générer pour correspondre à cette vente.
La situation du stock est alors mise à jour automatiquement.


\subsection{Impression}
\label{\detokenize{invoice/create_bill:impression}}
Depuis la fiche d’un document (devis, facture, reçu ou avoir) vous pouvez à tout moment imprimer ou ré-impriment celui-ci s’il n’est pas à l’état «en cours».


\subsection{Paiement}
\label{\detokenize{invoice/create_bill:paiement}}
Si ceux-ci sont configurés (menu « Administration/Configuration du règlement »), vous pouvez consulter les moyens de paiement d’une facture, d’un reçu ou d’un devis.
Si vous l’envoyez par courriel, vous pouvez également les faire apparaitre dans votre message.

Dans le cas d’un paiement via PayPal, si votre \_Diacamma\_ est accessible par internet, le logiciel sera automatiquement notifié du règlement.
Dans le cas d’un devis, celui-ci sera automatiquement archivé et une facture équivalente sera générée.
Un nouveau réglement sera ajouté dans votre facture.

Dans l’écran « Financier/Transactions bancaires », vous pouvez consulté précisement la notification reçu de PayPal.
En cas d’état « échec », la raison est alors précisé: il vous faudra manuellement vérifier votre compte PayPal et rétablir l’éventuellement paiment erroné manuellement.


\section{Gestion de stock}
\label{\detokenize{invoice/stock:gestion-de-stock}}\label{\detokenize{invoice/stock::doc}}

\subsection{Bordereau de stockage}
\label{\detokenize{invoice/stock:bordereau-de-stockage}}
Depuis le menu \sphinxstyleemphasis{facturier/Stockage/Bordereau de stockage} vous pouvez créer des bordereaux de reception ou sortie d’article stockable.

Pour cela, vous devez préciser un lieu de stockage qui sera impacté par le mouvement de stock (configurable depuis \sphinxstyleemphasis{Administration/Modules (conf.)/Configuration du facturier})
Dans le cas d’une réception, vous pouvez optionnellement préciser une référence de fournisseur (raison social, facture)
Ajoutez alors les articles ainsi que la valeur du mouvement.
Pour un borderaux de sortie, il n’est pas possible de saisir un montant suppérieur au stock courrant.


\subsection{Situation}
\label{\detokenize{invoice/stock:situation}}
Depuis le menu \sphinxstyleemphasis{facturier/Stockage/Situation} vous pouvez consulter, pour chaque lieu de stockage, la quantité de chaque article géré.

Le bouton \sphinxstyleemphasis{imprimer} permet de sortir un rapport au format PDF ou CSV (importable via un tableur)

Depuis la fiche d’un article, vous pouvez également consulter sa situation de stockage pour chacun des lieux de stockage défini.


\subsection{Historique}
\label{\detokenize{invoice/stock:historique}}
Depuis le menu \sphinxstyleemphasis{facturier/Stockage/Situation} vous pouvez consulter, le mouvement des articles en réception et en sortie, pour une période de temps données.

Le bouton \sphinxstyleemphasis{imprimer} permet également le même type de rapport que précédement.


\section{Configuration et paramétrage}
\label{\detokenize{invoice/configuration:configuration-et-parametrage}}\label{\detokenize{invoice/configuration::doc}}

\subsection{Catégories}
\label{\detokenize{invoice/configuration:categories}}
Le menu \sphinxstyleemphasis{Administration/Modules (conf.)/Configuration commercial du facturier}, onglet \sphinxstyleemphasis{Catégorie}

Vous pouvez définir des catégories afin de classer vos articles.
Chaque article peut être associé à plusieurs catégories.


\subsection{Champ personnalisé}
\label{\detokenize{invoice/configuration:champ-personnalise}}
Le menu \sphinxstyleemphasis{Administration/Modules (conf.)/Configuration commercial du facturier}, onglet \sphinxstyleemphasis{Champ personnalisé}

Comme pour les contacts, vous pouvez ici définir des champs personnalisés.


\subsection{Lieu de stockage}
\label{\detokenize{invoice/configuration:lieu-de-stockage}}
Le menu \sphinxstyleemphasis{Administration/Modules (conf.)/Configuration commercial du facturier}, onglet \sphinxstyleemphasis{Lieu de stockage}

Si vous voulez gérer une centrale d’achat, vous pouvez ici définir les différents espace de vos articles stockables.


\subsection{Réduction automatique}
\label{\detokenize{invoice/configuration:reduction-automatique}}
Le menu \sphinxstyleemphasis{Administration/Modules (conf.)/Configuration commercial du facturier}, onglet \sphinxstyleemphasis{Réduction automatique}
\begin{description}
\item[{Ce tableau de gestion de réductions comporte les champs suivants:}] \leavevmode\begin{itemize}
\item {} 
La catégorie d’article impacté.

\item {} 
Le type de réduction: en valeur, en pourcentage, en pourcentage global déjà vendu.

\item {} 
Le montant de la réduction (en valeur ou pourcentage suivant le type).

\item {} 
Le nombre d’occurrence déclenchant la réduction.

\item {} 
Un critère de filtrage du tiers à qui cette réduction s’applique. (Optionnel)

\end{itemize}

\end{description}

Au moment d’ajout d’un article dans une facture, si le client de cette facture et ce nouvel article répond à aux critères d’une réduction,
celle-ci s’applique alors automatiquement dans la facture.
Si plusieurs réductions remplissent leurs conditions, c’est la réduction octroyant la plus grande réduction qui sera utilisé.
Un article peut se retrouver vendu gratuitement, mais jamais négativement (qui reviendrait à un remboursement)
Ce mécanisme sera également appliqué lors de la création automatique des factures (cotisation, participation à un événement)
Ce mécanisme vérifie le critère que pour des opérations réalisés sur l’exercice financier courant de l’association (les réductions ne se cumule pas d’une année à l’autre)


\subsection{Codes d’imputations comptable}
\label{\detokenize{invoice/configuration:codes-d-imputations-comptable}}
Menu \sphinxstyleemphasis{Administration/Modules (conf.)/Configuration financière du facturier}, onglet \sphinxstyleemphasis{Codes d’imputations comptable}
\begin{description}
\item[{Un « Code d’imputation comptable » contiens:}] \leavevmode\begin{itemize}
\item {} 
un code comptable de vente

\item {} 
un code analytique (optionnel)

\end{itemize}

\end{description}

Chaque article peut être associé à un code d’imputation comptable (si non précisé, l’article n’est pas vendable).
Ce mécanisme permet de centraliser à un seul endroit les configurations comptables des articles.
Au changement d’exercice, si ces configurations doivent changées, il est plus simple de modifier cette configuration que l’ensemble des articles.


\subsection{Codes comptables par défaut}
\label{\detokenize{invoice/configuration:codes-comptables-par-defaut}}
Menu \sphinxstyleemphasis{Administration/Modules (conf.)/Configuration financière du facturier}, onglet \sphinxstyleemphasis{Paramètres}

Ce module est intimement lié au module de gestion comptable, un certain nombre de codes comptables par défaut sont nécessaires.

Pour pouvoir générer les écritures comptables correspondants aux factures saisies avec des articles non référencés, vous devez préciser le code comptable de vente (classe 7) lié a ce type de d’article. Par défaut, le code comptable de vente de service est défini.

Pour réaliser une réduction sur un article, vous devez préciser le code comptable de vente à imputer de cette réduction.
Dans le cas de règlement en liquide, il vous faut préciser le code comptable de banque associé à votre caisse.


\subsection{La configuration de la TVA}
\label{\detokenize{invoice/configuration:la-configuration-de-la-tva}}
Menu \sphinxstyleemphasis{Administration/Modules (conf.)/Configuration financière du facturier}, onglet \sphinxstyleemphasis{TVA}

Vous pouvez complètement configurer la gestion de votre soumission à la TVA.

\noindent\sphinxincludegraphics{{vat}.png}

Pour commencer, vous devez définir les modalités de soumission en sélectionnant votre mode d’application:
\begin{itemize}
\item {} \begin{description}
\item[{TVA non applicable}] \leavevmode
Vous n’êtes pas soumis à la TVA. L’ensemble de vos factures sont réalisées hors-taxe.

\end{description}

\item {} \begin{description}
\item[{Prix HT}] \leavevmode
Vous êtes soumis à la TVA. Vous faites le choix d’éditer vos factures avec les montants des articles en hors-taxe.

\end{description}

\item {} \begin{description}
\item[{Prix TTC}] \leavevmode
Vous êtes soumis à la TVA. Vous faites le choix d’éditer vos factures avec les montants des articles toutes taxes comprises.

\end{description}

\end{itemize}

Précisez également le compte comptable d’imputation de ces taxes.

Définissez l’ensemble des taux de taxes auxquels vos ventes sont soumises. La taxe par défaut sera celle appliquée à l’article libre (sans référence).


\chapter{Diacamma règlement}
\label{\detokenize{payoff/index:diacamma-reglement}}\label{\detokenize{payoff/index::doc}}
Aide relative aux fonctionnalités de gestion des payements.


\section{Règlement}
\label{\detokenize{payoff/payoff:reglement}}\label{\detokenize{payoff/payoff::doc}}
Depuis un module tel que la facturation, il vous est possible de gérer leur règlement.

Depuis la fiche du document, cliquez sur «ajouter» un paiement.
\begin{quote}

\noindent\sphinxincludegraphics{{payoff}.png}
\end{quote}

Vous pouvez alors saisir le mode de paiement de votre client ainsi que le compte bancaire à imputer de ce mouvement financier.

Dans la facture, vous pouvez consulter en plus de son montant total, la somme versée ainsi que le résidu de paiement à effectuer.

Chaque règlement génère automatiquement une écriture comptable dans le journal d’encaissement.

Il est aussi possible d’effectuer un seul règlement sur plusieurs document financier (comme les factures). Pour cela sélectionnez dans la liste des éléments « valides » celles que vous souhaitez et cliquez sur Réglement.
\begin{quote}

\noindent\sphinxincludegraphics{{multi-payoff}.png}
\end{quote}

Suivant le type de document sur lequel ce paiement est associé, vous pouvez avoir plusieurs modes de répartition:
\begin{itemize}
\item {} 
Par date
Ce paiement est d’abort ventilé sur le document financier le plus ancien, puis le suivant, etc.

\item {} 
Par prorata
Ce paiement multiple sera automatique ventilé sur document financier au prorata de leur montant.

\end{itemize}

Dans tout les cas, une seule écriture comptable d’encaissement sera alors réalisée.


\section{Dépôt de chèques}
\label{\detokenize{payoff/deposit:depot-de-cheques}}\label{\detokenize{payoff/deposit::doc}}
Depuis le menu \sphinxstyleemphasis{Finance/Dépôt de chèques}, vous pouvez ajouter et consulter des bordereaux de chèques.
\begin{quote}

\noindent\sphinxincludegraphics{{depositlist}.png}
\end{quote}

Dans une fiche de bordereau, vous pouvez sélectionner les règlements effectués par chèques dans vos différentes factures.
Cela vous constitue une liasse de chèques que vous pourrez déposer à votre agence bancaire.
Une fois réalisée, clôturez la sélection définie.
\begin{quote}

\noindent\sphinxincludegraphics{{deposititem}.png}
\end{quote}

Vous pouvez alors imprimer le bordereau de remise de chèques que vous pouvez joindre à votre liasse lors du dépôt de celle-ci.
De plus, une fois que votre bordereau apparaît sur votre relevé de compte, vous pouvez valider l’ensemble de vos écritures comptables depuis la fiche elle-même.


\section{Configuration}
\label{\detokenize{payoff/config:configuration}}\label{\detokenize{payoff/config::doc}}
Le menu \sphinxstyleemphasis{Administration/Configuration du règlement} vous permet quelques configurations pour votre structure.


\subsection{Compte bancaire}
\label{\detokenize{payoff/config:compte-bancaire}}
Dans cet écran, vous avez la possibilité d’enregistrer vos différents comptes bancaires que vous possédez.
Pour chacun, vous pouvez saisir l’intégralité des informations figurant sur un RIB.
Cela vous permettra d’éditer un résumé complet de vos dépôts de chèques.


\subsection{Moyen de paiement}
\label{\detokenize{payoff/config:moyen-de-paiement}}
Vous pouvez ici préciser les moyens de paiement que vous supportez.
Actuellement, 3 moyens de paiement sont pris en compte par \sphinxstyleemphasis{Diacamma}
\begin{itemize}
\item {} 
Le virement bancaire

\item {} 
Le chèque

\item {} 
Le paiement PayPal

\end{itemize}

Pour chacun d’entre eux, vous devez préciser les paramètres relatifs.

C’est moyen de paiement peuvent être utilisé pour vos débiteurs afin de régler par un de ses moyens ce qu’ils vous doivent.

Dans le cas de PayPal, si votre \sphinxstyleemphasis{Diacamma} est accessible par internet, le logiciel peux être notifié directement du paiement et ajouter un réglement correspondant directement dans votre logiciel.
Il est conseillé, dans ce cas, de cocher le champ \sphinxstyleemphasis{avec contrôle}: le lien de paiement présenter dans un courriel redirigera alors en premier sur votre \sphinxstyleemphasis{Diacamma} afin de vérifier que cet élément financier est toujours d’actualité.


\subsection{Paramètres}
\label{\detokenize{payoff/config:parametres}}
2 Paramètres actuellements:
\begin{itemize}
\item {} 
compte de caisse: indique le code comptable à imputer pour les règlements en espèce.

\item {} 
compte de frais bancaire: prècise un code comptable pour imputer directement, suite à un règlement, des frais bancaires inhérent à ce règlement.

\end{itemize}

Un ligne d’écriture est alors ajouté directement à l’écriture comptable correspondant.
Si ce code est vide, aucun frais bancaire ne vous sera demandé.


\chapter{Lucterios contacts}
\label{\detokenize{contacts/index:lucterios-contacts}}\label{\detokenize{contacts/index::doc}}
Aide relative aux fonctionnalités de gestion de contacts moraux ou physiques.


\section{Les contact physiques}
\label{\detokenize{contacts/individual:les-contact-physiques}}\label{\detokenize{contacts/individual::doc}}
Un contact physique est une personne, homme ou femme, à mémoriser.


\subsection{Liste de vos contacts physiques}
\label{\detokenize{contacts/individual:liste-de-vos-contacts-physiques}}
Le menu \sphinxstyleemphasis{Bureautique/Adresses et Contacts/Personnes Physiques} vous permet de consulter la liste des personnes que vous avez déjà enregistrées. Étant donné que la liste peut devenir importante, il est possible de filtrer les personnes par leur nom.

Depuis cet écran, vous avez aussi la possibilité d’imprimer la liste des personnes.

\noindent\sphinxincludegraphics{{ListIndividual}.png}


\subsection{Edition d’un contact physique}
\label{\detokenize{contacts/individual:edition-d-un-contact-physique}}
Depuis la liste précédente, vous avez la possibilité d’ajouter une nouvelle personne. Vous pouvez ré-éditer cette fiche depuis sa visualisation.

\noindent\sphinxincludegraphics{{EditIndividual}.png}


\subsection{Visualisation d’un contact physique}
\label{\detokenize{contacts/individual:visualisation-d-un-contact-physique}}
Depuis la liste des personnes physiques, vous avez la possibilité de visualiser une personne.

Cela vous permettra de consulter la fiche d’une personne précédemment enregistrée dans votre base. Vous pouvez modifier cette fiche ou l’imprimer. Vous pouvez également lui donner un alias de connexion à l’application associé à un droit d’accès (voir Les utilisateurs). Si cette personne n’est pas référencée dans d’autres enregistrements de l’application, vous avez la possibilité de la supprimer.

\noindent\sphinxincludegraphics{{ShowIndividual}.png}


\subsection{Recherche d’un contact physique}
\label{\detokenize{contacts/individual:recherche-d-un-contact-physique}}
Le menu Bureautique/Adresses et Contacts/Recherche de personne physique de personne physique vous permet de définir un critère de recherche sur une personne physique.

Une fois validé, l’outil va rechercher dans la base toutes les personnes correspondantes à ces critères. Vous pourrez alors imprimer cette liste ou en visualiser/modifier une fiche.

\noindent\sphinxincludegraphics{{FindIndividual}.png}


\section{Les contacts moraux}
\label{\detokenize{contacts/legal_entity:les-contacts-moraux}}\label{\detokenize{contacts/legal_entity::doc}}
Un contact moral est une structure ou d’une organisation de personne (entreprise, association, administration, …), à mémoriser.


\subsection{Liste de vos contacts moraux}
\label{\detokenize{contacts/legal_entity:liste-de-vos-contacts-moraux}}
Le menu \sphinxstyleemphasis{Bureautique/Adresses et Contacts/Personnes morales} vous permet de consulter la liste des structures que vous avez déjà enregistrées. Chaque contact moral est associé à une catégorie. Dans cette liste, vous consultez vos structures filtrées par ces catégories.

Depuis cet écran, vous avez aussi la possibilité d’imprimer la liste des structures.

\noindent\sphinxincludegraphics{{ListLegalEntity}.png}


\subsection{Edition d’un contact moral}
\label{\detokenize{contacts/legal_entity:edition-d-un-contact-moral}}
Depuis la liste précédente, vous avez la possibilité de créer une nouvelle structure. Vous pouvez ré-éditer cette fiche depuis sa visualisation.

\noindent\sphinxincludegraphics{{EditLegalEntity}.png}


\subsection{Visualisation d’un contact moral}
\label{\detokenize{contacts/legal_entity:visualisation-d-un-contact-moral}}
Depuis la liste des personnes morales, vous avez la possibilité de visualiser une structure.

Cela vous permettra de consulter la fiche d’une structure précédemment entrée dans votre base. Vous pouvez modifier cette fiche ou l’imprimer. Si cette personne n’est pas référencée dans d’autre enregistrement de l’application, vous avez la possibilité de la supprimer.

\noindent\sphinxincludegraphics{{ShowLegalEntity}.png}


\subsection{Responsables d’un contact moral}
\label{\detokenize{contacts/legal_entity:responsables-d-un-contact-moral}}
Vous avez la possibilité d’associer une personne physique à votre structure.

Choisissez le nouveau responsable: si la personne n’existe pas dans votre base, vous aurez la possibilité de la créer. Vous pourrez également ajouter une fonction à un responsable défini.

\noindent\sphinxincludegraphics{{ResponsabilityLegalEntity}.png}


\subsection{Recherche d’un contact moral}
\label{\detokenize{contacts/legal_entity:recherche-d-un-contact-moral}}
Le menu \sphinxstyleemphasis{Bureautique/Adresses et Contacts/Recherche de personne morale} vous permet de définir un critère de recherche sur une structure morale.

\noindent\sphinxincludegraphics{{FindLegalEntity}.png}


\section{Configuration et paramétrage}
\label{\detokenize{contacts/configuration:configuration-et-parametrage}}\label{\detokenize{contacts/configuration::doc}}
Dans le menu \sphinxstyleemphasis{Administration/Modules (conf.)} vous avez à votre disposition des outils pour configurer la gestion des contacts.


\subsection{Configuration des contacts}
\label{\detokenize{contacts/configuration:configuration-des-contacts}}
Dans cet écran, vous avez la possibilité de créer ou de modifier une définition de fonction, ou responsabilité, pour associer une personne physique à une structure morale. Vous pouvez créer ou modifier une catégorie de structure morale pour vous aider dans la classification de vos contacts moraux.

Il se peut que vous ayez besoin de préciser des informations supplémentaires pour vos différents contacts. Vous avez ici la possibilité d’ajouter des champs personnels pour chaque type de contacts. Pour ajouter un champ, vous devez simplement donner son titre ainsi que définir son type et éventuellement les compléments nécessaires.
5 types are possibles:
\begin{itemize}
\item {} 
chaîne de texte

\item {} 
nombre entier

\item {} 
nombre à virgule (réel)

\item {} 
valeur Oui/Non (booléen)

\item {} 
choix dans une liste (énumération)

\end{itemize}

Dans le cas de l’énumération, vous devez définir la liste des valeurs possibles (mots) séparées par un point-virgule.


\subsection{Codes postaux/villes}
\label{\detokenize{contacts/configuration:codes-postaux-villes}}
Cela peux vous aider dans votre saisi de contact, l’outil va automatiquement rechercher la ville (ou les villes) associée(s) avec le code postal que vous entrerez.
Dans cet écran, vous pouvez ajouter des codes postaux manquants.
Par défaut, les codes postaux français et suisses sont insérés.


\chapter{Lucterios courier}
\label{\detokenize{mailing/index:lucterios-courier}}\label{\detokenize{mailing/index::doc}}
Aide relative aux fonctionnalités de courier et publipostage.


\section{Configuration du couriel}
\label{\detokenize{mailing/configuration:configuration-du-couriel}}\label{\detokenize{mailing/configuration::doc}}
Vous pouvez configurer ici des réglages pour l’envoi de couriel.

Le serveur SMTP permettra au logiciel d’envoyer directement des messages à vos contacts.
Configurez donc ici les règlages de votre serveur.
Vous pouvez également préciser un \sphinxstyleemphasis{Fichier privé DKIM} et \sphinxstyleemphasis{Sélecteur DKIM} afin de signer vos envoies de courriel.
Les paramètres \sphinxstyleemphasis{durée (en min) d’un lot de courriel} et \sphinxstyleemphasis{nombre de courriels par lot} sont utilisés pour l’envoie des messages en publipostage.

Un bouton \sphinxstyleemphasis{Envoyer} permet de tester vos règlages en envoyant un courriel de test à un destinataire choisi.
Il existe des outils permettant de vérifier si vos messages respectent des règles afin d’éviter d’être considérer comme des “pourriel”.
En autre, l’outil \sphinxurl{https://www.mail-tester.com} (gratuit jusqu’à 3 fois par jour) vous permet, en envoyant un message à l’adresse précisée, de vous établir une note de confiance.

Vous pouvez, entre autre, envoyer d’un nouveau mot de passe de connexion.
N’oubliez pas alors de préciser un petit message d’explication via le paramètres \sphinxstyleemphasis{Message de confirmation de connexion}.


\section{Publipostage}
\label{\detokenize{mailing/mailing:publipostage}}\label{\detokenize{mailing/mailing::doc}}
Depuis le menu \sphinxstyleemphasis{Bureatique/Publipostage/Message} vous avez la possibilité de créer un courier de publipostage.


\subsection{Création d’un message}
\label{\detokenize{mailing/mailing:creation-d-un-message}}
Une fois votre message rédigé, vous pouvez lui associé des requetes de destinataires.
C’est requetes de recherches, similaire à celle des outils de recherche de contacts, ne seront évaluées qu’au moment de la génération du courier.
Ainsi, même un contact dernièrement ajouté ou modifié pourra être impacté par ce message.

Il est également possible d’ajouter à votre message un ou plusieurs documents, sauvés dans le \sphinxstyleemphasis{gestionnaire de documentation}.
Ces documents seront transmis en pièces-jointes dans l’envoie par courriel.

L’option \sphinxstyleemphasis{document(s) ajouté(s) via liens dans le message} permet d’ajouter un ensemble de liens partagés vers vos documents (et non plus des pièces jointes).
Cela permet de gérer des documents de taille importante ou qui risqueraient d’être supprimer par certain gestionnaire de courriel.

\noindent\sphinxincludegraphics{{mailing}.png}


\subsection{Validation \& transmission}
\label{\detokenize{mailing/mailing:validation-transmission}}\begin{description}
\item[{Une fois le message validé vous pouvez:}] \leavevmode\begin{itemize}
\item {} 
Soit généré une sortie PDF de l’ensemble des lettres à envoyer personnalisé avec l’entête de chaque contact

\item {} 
Soit envoyé par courriel si votre configuration est valide. Bien sur, dans ce cas, seul les contacts possédant une adresse seront impacté par cet envoie.

\end{itemize}

\end{description}

De plus, dans le cas d’un envoie par courriel, vous pouvez consulter un rapport de transmission.
Celui-ci vous indique les courriels envoyés, leur éventuel erreur d’acheminement.

Si votre logiciel est accessible depuis internet, vous pouvez également consulter le nombre de fois que le destinataire à consulter ce message.
Ce mécanisme se base sur l’acceptation, par votre destinataire des images distantes présentent dans le message.

\noindent\sphinxincludegraphics{{transmission}.png}


\chapter{Lucterios documents}
\label{\detokenize{documents/index:lucterios-documents}}\label{\detokenize{documents/index::doc}}
Aide relative aux fonctionnalités de gestion documentaire.


\section{Fichiers partagés}
\label{\detokenize{documents/shared_document:fichiers-partages}}\label{\detokenize{documents/shared_document::doc}}

\subsection{Liste des documents}
\label{\detokenize{documents/shared_document:liste-des-documents}}
Le menu \sphinxstyleemphasis{Bureautique/Gestion documentaire/Documents} vous permet de consulter la liste des fichiers que vous avez déjà enregistrés. Pour vous aider à retrouver vos documents, la liste est classifiée par un ensemble de dossiers et de sous-dossiers et une description vous donne un petit résumé.

Vous avez aussi la possibilité d’ajouter un sous-dossier ou de modifier les propriétés du dossier courant.

\noindent\sphinxincludegraphics{{listdoc}.png}

Suivant vos permissions, vous pouvez extraire votre fichier pour le consulter, le modifier et éventuellement ré-injecter vos corrections.

De plus, l’outil mémorisera l’utilisateur et la date de création du document ainsi que les informations relatives à la dernière modification.

\noindent\sphinxincludegraphics{{showdoc}.png}

Depuis la fiche du document, il vous est possible d’activer un lien de téléchargement.
Ce lien web peut être transmis à une personne tiers, n’ayant aucun droit d’accès à votre logiciel, afin de télécharger le document.
\sphinxstylestrong{Attention:} Votre instance doit être accessible sur internet pour que le lien puisse fonctionner depuis n’importe où.


\subsection{Recherche de documents}
\label{\detokenize{documents/shared_document:recherche-de-documents}}
Le menu \sphinxstyleemphasis{Bureautique/Gestion documentaire/Recherche de document} vous permet de définir un critère de recherche sur un document.

Une fois validé, l’outil va rechercher dans la base toutes les fichiers correspondants à ces critères.


\section{Configuration}
\label{\detokenize{documents/configuration:configuration}}\label{\detokenize{documents/configuration::doc}}
Dans le menu \sphinxstyleemphasis{Administration/Module (conf)/Dossiers} vous avez à votre disposition un ensemble d’outils pour configurer la gestion documentaire.


\subsection{Dossiers}
\label{\detokenize{documents/configuration:dossiers}}
Dans cet écran vous avez la possibilité de créer ou de modifier des dossiers de classement documentaire.

\noindent\sphinxincludegraphics{{configuration}.png}

En associant judicieusement un dossier comme sous-dossier d’un parent, vous pouvez vous définir une arborescence de classement.

Vous associez à chaque dossier un ensemble de groupe de droits pour la visualisation et la modification des fichiers. Seuls les utilisateurs appartenant aux groupes de visualisation pourront consulter les documents de cette catégorie. Seuls les utilisateurs appartenant aux groupes de modification pourront corriger les documents de cette catégorie.


\section{Editeur de documents}
\label{\detokenize{documents/editor:editeur-de-documents}}\label{\detokenize{documents/editor::doc}}
Il est possible de configurer l’outil afin de pouvoir éditer certain document directement via l’interface « en ligne ».

Des outils d’édition, libres et gratuits, sont actuellement configurable afin de les utiliser pour consulter et modifier des documents.
\_Note:\_ Ces outils sont gérés par des équipes complètement différentes, il se peux que certain de leur comportement ne correspondent pas à vos attentes.


\subsection{etherpad}
\label{\detokenize{documents/editor:etherpad}}
Editeur pour document textuel.
\begin{description}
\item[{Site Web}] \leavevmode
\sphinxurl{https://etherpad.org/}

\item[{Installation}] \leavevmode
Le tutoriel de framasoft explique bien comment l’installer
\sphinxurl{https://framacloud.org/fr/cultiver-son-jardin/etherpad.html}

\item[{Configurer}] \leavevmode
Editer le fichier « settings.py » contenu dans le répertoire de votre instance.
Ajouter et adapter la ligne ci-dessous:
\begin{itemize}
\item {} 
url : adresse d’accès d’etherpad

\item {} 
apikey : contenu de la clef de sécurité (fichier APIKEY.txt contenu dans l’installation d’etherpad)

\end{itemize}

\end{description}

\begin{sphinxVerbatim}[commandchars=\\\{\}]
\PYG{c+c1}{\PYGZsh{} extra}
\PYG{n}{ETHERPAD} \PYG{o}{=} \PYG{p}{\PYGZob{}}\PYG{l+s+s1}{\PYGZsq{}}\PYG{l+s+s1}{url}\PYG{l+s+s1}{\PYGZsq{}}\PYG{p}{:} \PYG{l+s+s1}{\PYGZsq{}}\PYG{l+s+s1}{http://localhost:9001}\PYG{l+s+s1}{\PYGZsq{}}\PYG{p}{,} \PYG{l+s+s1}{\PYGZsq{}}\PYG{l+s+s1}{apikey}\PYG{l+s+s1}{\PYGZsq{}}\PYG{p}{:} \PYG{l+s+s1}{\PYGZsq{}}\PYG{l+s+s1}{jfks5dsdS65lfGHsdSDQ4fsdDG4lklsdq6Gfs4Gsdfos8fs}\PYG{l+s+s1}{\PYGZsq{}}\PYG{p}{\PYGZcb{}}
\end{sphinxVerbatim}
\begin{description}
\item[{Usage}] \leavevmode\begin{description}
\item[{Dans le gestionnaire de document, vous avez plusieurs action qui apparait alors}] \leavevmode\begin{itemize}
\item {} 
Un bouton « + Fichier » vous permettant de créer un document txt ou html

\item {} 
Un bouton « Editeur » pour ouvrir l’éditeur etherpad.

\end{itemize}

\end{description}

\end{description}

\noindent\sphinxincludegraphics{{etherpad}.png}


\subsection{ethercalc}
\label{\detokenize{documents/editor:ethercalc}}
Editeur pour tableau de calcul.
\begin{description}
\item[{Site Web}] \leavevmode
\sphinxurl{https://ethercalc.net/}

\item[{Installation}] \leavevmode
Sur le site de l’éditeur, une petit explication indique comment l’installer.

\item[{Configurer}] \leavevmode
Editer le fichier « settings.py » contenu dans le répertoire de votre instance.
Ajouter et adapter la ligne ci-dessous:
\begin{itemize}
\item {} 
url : adresse d’accès d’ethercal

\end{itemize}

\end{description}

\begin{sphinxVerbatim}[commandchars=\\\{\}]
\PYG{c+c1}{\PYGZsh{} extra}
\PYG{n}{ETHERCALC} \PYG{o}{=} \PYG{p}{\PYGZob{}}\PYG{l+s+s1}{\PYGZsq{}}\PYG{l+s+s1}{url}\PYG{l+s+s1}{\PYGZsq{}}\PYG{p}{:} \PYG{l+s+s1}{\PYGZsq{}}\PYG{l+s+s1}{http://localhost:8000}\PYG{l+s+s1}{\PYGZsq{}}\PYG{p}{\PYGZcb{}}
\end{sphinxVerbatim}
\begin{description}
\item[{Usage}] \leavevmode\begin{description}
\item[{Dans le gestionnaire de document, vous avez plusieurs action qui apparait alors}] \leavevmode\begin{itemize}
\item {} 
Un bouton « + Fichier » vous permettant de créer un document csv, ods ou xmlx

\item {} 
Un bouton « Editeur » pour ouvrir l’éditeur ethercalc.

\end{itemize}

\end{description}

\end{description}

\noindent\sphinxincludegraphics{{ethercalc}.png}


\chapter{Coeur Lucterios}
\label{\detokenize{CORE/index:coeur-lucterios}}\label{\detokenize{CORE/index::doc}}
Aide relative aux fonctionnalités générales de cet outil de gestion.


\section{Mot de passe}
\label{\detokenize{CORE/password:mot-de-passe}}\label{\detokenize{CORE/password::doc}}
Le menu \sphinxtitleref{Général/Mot de passe} vous permet de changer le mot de passe d’accès de l’utilisateur courant.

\noindent\sphinxincludegraphics{{password}.png}

Pour plus de sécurité, nous vous conseillons d’utiliser un mot de passe comprenant des lettres et des chiffres et ne constituant pas un mot compréhensible.


\section{Les groupes}
\label{\detokenize{CORE/groups:les-groupes}}\label{\detokenize{CORE/groups::doc}}
Le menu \sphinxtitleref{Administration/Gestion des Droits/Les groupes} vous permet de créer, modifier ou supprimer un groupe de droits.

\noindent\sphinxincludegraphics{{group}.png}

Un groupe de droits réunit un ensemble d’autorisations aux actions de l’application.

\noindent\sphinxincludegraphics{{group_modify}.png}


\section{Les utilisateurs}
\label{\detokenize{CORE/users:les-utilisateurs}}\label{\detokenize{CORE/users::doc}}
Le menu \sphinxtitleref{Administration/Gestion des Droits/Les utilisateurs} vous
permet de créer, modifier ou désactiver un utilisateur de l’application. Un
utilisateur définit un droit de connexion au logiciel.

\noindent\sphinxincludegraphics{{users}.png}

Depuis cette liste, vous pouvez créer ou modifier l’utilisateur: son
alias, son nom et son mot de passe. A cela, vous lui ajouter des groupes et
des permissions suplémentaires éventuelles afin de définir son niveau
d’accès au logiciel. Vous pouvez aussi désactiver un utilisateur pour lui
interdire l’accès à l’application.

\noindent\sphinxincludegraphics{{user_info}.png}

\noindent\sphinxincludegraphics{{user_permissions}.png}


\section{L’architecture du logiciel}
\label{\detokenize{CORE/architecture:l-architecture-du-logiciel}}\label{\detokenize{CORE/architecture::doc}}
Depuis le commencement de ce logiciel, les développeurs ont voulu que cette application puisse avoir une architecture ouverte permettant des évolutions les plus larges.



\renewcommand{\indexname}{Index}
\printindex
\end{document}